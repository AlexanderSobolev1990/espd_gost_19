\annotation

Данный документ является примером оформления текста с использованием системы верстки (La)\TeX. Ссылка: \url{https://en.wikipedia.org/wiki/LaTeX}. Отличительной чертой проекта, намного повышающей удобсто использования, является использование файла UseLatex.cmake, который позволяет легко и просто собирать исходные тексты из *.tex файлов путем написания соответствующего CMakeLists.txt (пример имеется в директории проекта) и вызова процесса сборки стандартным способом: \lstinline|mkdir build && cd build && cmake .. && make|.

Доработанный класс espd.cls позволяет легко и просто оформлять титульную страницу и лист утверждения по ГОСТ-19, а также включает все необходимое оформление. Таким образом, использование данного класса и языка разметки (La)\TeX позволяет техническому писателю сконцентрироваться на главном\mdash написании текста. Оформление формул, таблиц, вставка рисунков также значительно упрощаются, исключается их <<съезжание>>, как часто случается при исползовании текстового редактора Word.

Далее изложены наиболее часто встречающиеся конструкции, необходимые для написания текста технического задания и остальной документации по ГОСТ\sdash 19.

%\begin{lstlisting}
%namespace SPML /// Специальная библиотека программных модулей (СБ ПМ)
%{
%namespace Compare /// Сравнение чисел
%{
%static const float EPS_F = 1.0e-4f; ///< Абсолютная точность по умолчанию при сравнениях чисел типа float (1.0e-4)
%static const double EPS_D = 1.0e-8; ///< Абсолютная точность по умолчанию при сравнениях чисел типа double (1.0e-8)
%static const float EPS_REL = 0.01; ///< Относительная точность по умолчанию
%
%///
%/// \brief Сравнение двух действительных чисел (по абсолютной разнице)
%/// \details Возвращает результат: abs( first - second ) < eps
%/// \param[in] first  - первое число
%/// \param[in] second - второе число
%/// \param[in] eps - абсолютная точность сравнения
%/// \return true - если разница меньше точности, иначе false
%///
%inline bool AreEqualAbs( float first, float second, const float &eps = EPS_F )
%{
%return ( std::abs( first - second ) <= eps );
%}
%
%BOOST_AUTO_TEST_CASE( test_mat_1_seqextr_min )
%{
%int size = mat_1_dense.n_cols;
%arma::ivec actual = arma::ivec( size, arma::fill::zeros );
%double infValue = 1e7;//mat_1_dense.max();
%double resolution = 1e-7;
%double lapcost;
%SPML::LAP::SequentalExtremum( mat_1_dense, SPML::LAP::TSearchParam::SP_Min, infValue, resolution, actual, lapcost );
%double eps = 1e-7;
%BOOST_CHECK_EQUAL( arma::approx_equal( actual, expected_1_subextr_min, "absdiff", eps ), true );
%
%arma::ivec actualCOO = arma::ivec( size, arma::fill::zeros );
%double lapcostCOO;
%SPML::Sparse::CMatrixCOO coo;
%SPML::Sparse::MatrixDenseToCOO( mat_1_dense, coo );
%SPML::LAP::SequentalExtremum( coo, SPML::LAP::TSearchParam::SP_Min, infValue, resolution, actualCOO, lapcostCOO );
%BOOST_CHECK_EQUAL( arma::approx_equal( actualCOO, expected_1_subextr_min, "absdiff", eps ), true );
%}
%\end{lstlisting}
