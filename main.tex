%%%%%%%%%%%%%%%%%%%%%%%%%%%%%%%%%%%%%%%%%%%%%%%%%%%%%%%%%%%%%%%%%%%%%%%%%%%%%%%%
%
% \file       main.tex
% \brief      Главный файл настроек TeX проекта
% \date       18.05.22 - создан
% \author     Соболев А.А.
%
%%%%%%%%%%%%%%%%%%%%%%%%%%%%%%%%%%%%%%%%%%%%%%%%%%%%%%%%%%%%%%%%%%%%%%%%%%%%%%%%
\documentclass[exampletask]{espd}
\usepackage[russian]{babel}
\usepackage{array}
\usepackage{setspace}
\usepackage{fontspec}
\usepackage{cite}
%\usepackage{amsmath}
\usepackage{gensymb} % Для знака градуса
\usepackage{mathtools}
\usepackage{listings} % Для листингов исходного кода
\usepackage{color}

\bibliographystyle{gost2008}
\setmainfont{Times New Roman}

%%%%%%%%%%%%%%%%%%%%%%%%%%%%%%%%%%%%%%%%%%%%%%%%%%%%%%%%%%%%%%%%%%%%%%%%%%%%%%%%
% Переопределение команд дефиса
%%%%%%%%%%%%%%%%%%%%%%%%%%%%%%%%%%%%%%%%%%%%%%%%%%%%%%%%%%%%%%%%%%%%%%%%%%%%%%%%
\newcommand{\sdash}{\nobreakdash-}  % Дефис неразрывный без пробелов до и после
\newcommand{\ndash}{\nobreakdash~--~}  % Короткое тире неразрывное с пробелами до и после
\newcommand{\mdash}{\nobreakdash~---~} % Длинное тире неразрывное с пробелами до и после
%%%%%%%%%%%%%%%%%%%%%%%%%%%%%%%%%%%%%%%%%%%%%%%%%%%%%%%%%%%%%%%%%%%%%%%%%%%%%%%%
% Определение форматирования исходного кода в тексте документа
%%%%%%%%%%%%%%%%%%%%%%%%%%%%%%%%%%%%%%%%%%%%%%%%%%%%%%%%%%%%%%%%%%%%%%%%%%%%%%%%
\definecolor{dkgreen}{rgb}{0,0.5,0}
\definecolor{gray}{rgb}{0.5,0.5,0.5}
\definecolor{mauve}{rgb}{0.58,0,0.82}
\lstset{
	frame=tb,
	language=C++,
	aboveskip=3mm,
	belowskip=3mm,
	showstringspaces=false,
	columns=flexible,
	basicstyle=\listingtextsize\fontspec{Courier New},
	extendedchars=true, % без символа \ !!! - так поддерживается русский язык в комментариях
	inputencoding=utf8x,
%    escapechar=|,
	numbers=none,
	numberstyle=\tiny\color{mauve},
	keywordstyle=\color{blue},
	commentstyle=\color{dkgreen},
	stringstyle=\color{mauve},
	breaklines=true,
	breakatwhitespace=true,
	tabsize=3
}
% Поддержка русского языка в комментариях к исходному коду
\makeatletter % see https://tex.stackexchange.com/a/320345
\lst@InputCatcodes
\def\lst@DefEC{%
	\lst@CCECUse \lst@ProcessLetter
	^^80^^81^^82^^83^^84^^85^^86^^87^^88^^89^^8a^^8b^^8c^^8d^^8e^^8f%
	^^90^^91^^92^^93^^94^^95^^96^^97^^98^^99^^9a^^9b^^9c^^9d^^9e^^9f%
	^^a0^^a1^^a2^^a3^^a4^^a5^^a6^^a7^^a8^^a9^^aa^^ab^^ac^^ad^^ae^^af%
	^^b0^^b1^^b2^^b3^^b4^^b5^^b6^^b7^^b8^^b9^^ba^^bb^^bc^^bd^^be^^bf%
	^^c0^^c1^^c2^^c3^^c4^^c5^^c6^^c7^^c8^^c9^^ca^^cb^^cc^^cd^^ce^^cf%
	^^d0^^d1^^d2^^d3^^d4^^d5^^d6^^d7^^d8^^d9^^da^^db^^dc^^dd^^de^^df%
	^^e0^^e1^^e2^^e3^^e4^^e5^^e6^^e7^^e8^^e9^^ea^^eb^^ec^^ed^^ee^^ef%
	^^f0^^f1^^f2^^f3^^f4^^f5^^f6^^f7^^f8^^f9^^fa^^fb^^fc^^fd^^fe^^ff%
	^^^^20ac^^^^0153^^^^0152%
	% Basic Cyrillic alphabet coverage
	^^^^0410^^^^0411^^^^0412^^^^0413^^^^0414^^^^0415^^^^0416^^^^0417%
	^^^^0418^^^^0419^^^^041a^^^^041b^^^^041c^^^^041d^^^^041e^^^^041f%
	^^^^0420^^^^0421^^^^0422^^^^0423^^^^0424^^^^0425^^^^0426^^^^0427%
	^^^^0428^^^^0429^^^^042a^^^^042b^^^^042c^^^^042d^^^^042e^^^^042f%
	^^^^0430^^^^0431^^^^0432^^^^0433^^^^0434^^^^0435^^^^0436^^^^0437%
	^^^^0438^^^^0439^^^^043a^^^^043b^^^^043c^^^^043d^^^^043e^^^^043f%
	^^^^0440^^^^0441^^^^0442^^^^0443^^^^0444^^^^0445^^^^0446^^^^0447%
	^^^^0448^^^^0449^^^^044a^^^^044b^^^^044c^^^^044d^^^^044e^^^^044f%
	^^^^0401^^^^0451^^^^0020%
	%%%
	^^00}
\lst@RestoreCatcodes
\makeatother
%%%%%%%%%%%%%%%%%%%%%%%%%%%%%%%%%%%%%%%%%%%%%%%%%%%%%%%%%%%%%%%%%%%%%%%%%%%%%%%%%
%% Задание перечислений буквами
%%%%%%%%%%%%%%%%%%%%%%%%%%%%%%%%%%%%%%%%%%%%%%%%%%%%%%%%%%%%%%%%%%%%%%%%%%%%%%%%%
%\usepackage{enumitem}
%
%\makeatletter
%\newcommand{\realasbuk}[1]{\expandafter\russian@realalph\csname c@#1\endcsname}
%
%\def\russian@realAlph#1{\ifcase#1\or
%	А\or Б\or В\or Г\or Д\or Е\or Ж\or
%	З\or И\or К\or Л\or М\or Н\or О\or
%	П\or Р\or С\or Т\or У\or Ф\or Х\or
%	Ц\or Ч\or Ш\or Щ\or Э\or Ю\or Я\else\xpg@ill@value{#1}{russian@Alph}\fi}
%\def\russian@realalph#1{\ifcase#1\or
%	а\or б\or в\or г\or д\or е\or ж\or
%	з\or и\or к\or л\or м\or н\or о\or
%	п\or р\or с\or т\or у\or ф\or х\or
%	ц\or ч\or ш\or щ\or э\or ю\or я\else\xpg@ill@value{#1}{russian@alph}\fi}
%
%\AddEnumerateCounter{\realasbuk}{\russian@realalph}{щ}
%\makeatother
%%%%%%%%%%%%%%%%%%%%%%%%%%%%%%%%%%%%%%%%%%%%%%%%%%%%%%%%%%%%%%%%%%%%%%%%%%%%%%%%
% Задание полей титульных страниц
%%%%%%%%%%%%%%%%%%%%%%%%%%%%%%%%%%%%%%%%%%%%%%%%%%%%%%%%%%%%%%%%%%%%%%%%%%%%%%%%
\newcommand{\productcipher}{АБВГД.12345} % Шифр изделия

\firstapplication{\productcipher} % Первое применение
\referencenumber{\productcipher} % Справ №

\organizationcode{01234} % Код организации
\registrationcode{56789} % Регистрационный код
\redaction{01} % Номер редакции
\documentnumber{03} % Номер документа данного вида
%\partnumber{1} % Номер части документа - чтобы задать - надо фиксить класс espd

% СОГЛАСОВАНО
\customerrank{Начальник\\межгалактической комиссии}
\customername{А.Б.~Заказчиков}

% УТВЕРЖДАЮ
\chiefconstructorrank{Главный конструктор\\изделия \productcipher}
\chiefconstructorname{А.Б.~Главный}

\fromcustomerrank{От межгалактической комиссии}
\fromcustomername{А.Б.~Галактионов}

\headofdepartmentrank{Начальник Центра}
\headofdepartmentname{А.Б.~Чатланин}

\deputyofchiefconstructorrank{Зам.~гл.~конструктора}
\deputyofchiefconstructorname{А.Б.~Заместителев}
%
\developerrank{Разработчик}
\developername{А.Б.~Разработчиков}

%\headoflaboratoryrank{Начальник лаборатории 777}
%\headoflaboratoryname{А.Б.~Лабораториев}

% Исполнитель(и)
\authorname{А.Б.~Пацак}
%%
%\authori{А.Б.~Пацак1}
%\authorii{А.Б.~Пацак2}

\normocontrollerrank{Нормоконтроллер}
\normocontrollername{~}

\title{Изделие \productcipher\\Программный комплекс\\Галактический транклюкатор} 
\year{2022}
%%%%%%%%%%%%%%%%%%%%%%%%%%%%%%%%%%%%%%%%%%%%%%%%%%%%%%%%%%%%%%%%%%%%%%%%%%%%%%%%
% Начало документа
%%%%%%%%%%%%%%%%%%%%%%%%%%%%%%%%%%%%%%%%%%%%%%%%%%%%%%%%%%%%%%%%%%%%%%%%%%%%%%%%
\begin{document}
\clubpenalty=10000  % Это костыль против
\widowpenalty=10000 % "висячих" строк
\righthyphenmin=200 % Избавляемся
{\sloppy	            % от переносов слов

\annotation

Пример текста для раздела аннотация супер-мега-важного примера оформления документа на галактический транклюкатор.

Далее изложены наиболее часто встречающиеся конструкции, необходимые для написания текста технического задания и остальной документации по ГОСТ19 % Аннотация
\tableofcontents % Содержание
% Дальше пошли разделы
%\include{all2} 
\section{Раздел}

В данном разделе приводится пример иерархии вложенности (ГОСТ 19.106 \cite{gost19_106}), которая включает в себя разделы, подразделы, пункты, подпункты и перечисления.

Подразделы, пункты и подпункты без заголовков не поддерживаются, но оно и к лучшему\mdash если не знаешь как озаглавить, то о чем собрался писать?

\subsection{Подраздел 1}
Текст текст текст текст текст текст текст текст текст текст текст текст текст текст текст текст текст текст текст текст текст текст текст текст текст текст текст.

\subsection{Подраздел 2}
Текст текст текст текст текст текст текст текст текст текст текст текст текст текст текст текст текст текст текст текст текст текст текст текст текст текст текст.

\paragraph{Пункт 1} 
Текст текст текст текст текст текст текст текст текст текст текст текст текст текст текст текст текст текст текст текст текст текст текст текст текст текст текст.

\paragraph{Пункт 2} 
Текст текст текст текст текст текст текст текст текст текст текст текст текст текст текст текст текст текст текст текст текст текст текст текст текст текст текст.

\subparagraph{Подпункт 1} 
Текст текст текст текст текст текст текст текст текст текст текст текст текст текст текст текст текст текст текст текст текст текст текст текст текст текст текст.

\subparagraph{Подпункт 2} 
Текст текст текст текст текст текст текст текст текст текст текст текст текст текст текст текст текст текст текст текст текст текст текст текст текст текст текст.

\subparagraph{Подпункт 3} 
Текст текст текст текст текст текст текст текст текст текст текст текст текст текст текст текст текст текст текст текст текст текст текст текст текст текст текст.

\subparagraph{Пример одноуровнего перечисления} 

\begin{enumerate}
\item пункт перечисления;
\item пункт перечисления;
\item пункт перечисления;
\end{enumerate}

\subparagraph{Пример вложенного перечисления} 

%\begin{enumi}
%	\item пункт первый с большим количеством слов
%	\item здесь будет подпункт с немалым количеством слов
%	\begin{itemize}
%		\item подпункт первый с немалым количеством слов
%		\begin{itemize}
%			\item подпункт второй вложенности
%			\item еще один, коротенький пункт
%		\end{itemize}
%		\item подпункт первой степени вложенности
%	\end{itemize}
%	\item пункт
%\end{enumi}
%
\begin{enumerate}
\item пункт перечисления;
\item пункт перечисления;	
\begin{enumerate}%[label={\realasbuk*}, ref=\realasbuk*]
	\item пункт вложенного перечисления;
	\item пункт вложенного перечисления;
	\item пункт вложенного перечисления;
\end{enumerate}	
\item пункт перечисления;
\end{enumerate}







 
%\section{Основание для разработки}

%Основанием для разработки является многократно обдуманное решение научиться писать техническую документацию в (xe)latex, чтобы конечный вид получаемого продукта (pdf-файла) не менялся как в, прости Господи, ворде при переходе на другую версию, чтобы абзацы не съезжали, чтобы стиль не портился от случайных кривостей разных версий ворда.
%
%Также, написание документации в (xe)latex позволяет использовать контроль версий git, раздельное написание глав (возможно несколькими авторами одновременно) в разных файлах. Также приходит понимание, что создается поистине ПРОДУКТ, а не вордовская писулька.
%
%\cite{espd005}

\subsection{Предварительный состав программной документации}\label{subsection:documentation}
«Кроссплатформенный облачный текстовый редактор "Notepad.Online"». Техническое задание (ГОСТ 19.201-78~\cite{espd201})

«Кроссплатформенный облачный текстовый редактор "Notepad.Online"». Программа и методика испытаний (ГОСТ 19.301-78~\cite{espd301})

«Кроссплатформенный облачный текстовый редактор "Notepad.Online"». Пояснительная записка (ГОСТ 19.404-79~\cite{espd404})

«Кроссплатформенный облачный текстовый редактор "Notepad.Online"». Руководство оператора (ГОСТ 19.505-79~\cite{espd505})

«Кроссплатформенный облачный текстовый редактор "Notepad.Online"». Текст программы (ГОСТ 19.401-78~\cite{espd401})

\subsection{Специальные требования к программной документации}
\begin{enumerate}
	\item Пояснительная записка должна быть загружена через информационно-образовательную среду НИУ ВШЭ LMS в систему «Антиплагиат», допустимый процент заимствования – 40\%;
	\item Техническая документация, программа, исходные коды и презентация загружаются в LMS одним архивом в формате .zip.
	\item Согласованная и утвержденная документация сдается в печатном виде в учебный офис образовательной программы 09.03.04 «Программная инженерия» НИУ ВШЭ.
\end{enumerate}

Сроки сдач всех указанных документов и данных определяются приказом декана Факультета компьютерных наук И.В. Аржанцева № 2.3-02/1012-0 2 от 10.12.2018. 
 
\bibliography{lit} % Список литературы
\begin{terms}
\term{Python}{интерпретируемый язык программирования высокого уровня}
\term{Десктопное приложение}{совокупность исполняемого файла, ресурсов и динамически загружаемых библиотек, которые содержат программу и предназначены для ее запуска в операционной системе Windows}
\term{Мобильное приложение}{программа, хранимая в виде файла с расширением .apk для запуска в операционной среде Android}
\term{Облачное хранилище}{сервисная модель, в которой данные хранятся, управляются, резервируются удаленно и предоставляются пользователям по сети}
\term{Окружение приложения}{контекст, в котором выполняется программа, т.е. операционная система и другие элементы, которые потенциально могут влиять на работу приложения}
\term{Отмена операций}{часть инструментального функционала большинства текстовых редакторов, функция, позволяющая отменить внесенные пользователем изменения в документ, такие как вставка символов, удаление символов, влияющие на содержимое встроенные функции программы}
\term{Плагин (расширение)}{дополнительный модуль программы, подключаемый к основной программе во время выполнения (Runtime) и предназначенный для расширения функционала основной программы}
\term{Прозрачное шифрование базы данных}{функция, выполняющая шифрование и дешифрование ввода-вывода в реальном времени для файлов данных и журналов. Ключ шифрования при этом хранится в загрузочной записи базы данных}
\term{Сниппет}{функция, заменяющая при определенных условиях подстроку в текстовом поле на другую строку, заранее заданную или сгенерированную по заранее определенным правилам}
\end{terms} % Список терминов
\begin{abbreviations}
\abbr{API}{Application Programming Interface}
\abbr{USB}{Universal Serial Bus}
\abbr{ПК}{персональный компьютер}
\abbr{ОС}{операционная система}
\end{abbreviations} % Список сокращений
\attachment{Справочное}{Окно программы <<Блокнот>> ОС Windows 10}
\image{notepad}
}
\end{document}
