\section{Раздел}

В данном разделе приводится пример иерархии вложенности (ГОСТ 19.106 \cite{gost19_106}), которая включает в себя разделы, подразделы, пункты, подпункты и перечисления.

Подразделы, пункты и подпункты без заголовков не поддерживаются, но оно и к лучшему\mdash если не знаешь как озаглавить, то о чем собрался писать?

\subsection{Подраздел 1}
Текст текст текст текст текст текст текст текст текст текст текст текст текст текст текст текст текст текст текст текст текст текст текст текст текст текст текст.

\subsection{Подраздел 2}
Текст текст текст текст текст текст текст текст текст текст текст текст текст текст текст текст текст текст текст текст текст текст текст текст текст текст текст.

\paragraph{Пункт 1} 
Текст текст текст текст текст текст текст текст текст текст текст текст текст текст текст текст текст текст текст текст текст текст текст текст текст текст текст.

\paragraph{Пункт 2} 
Текст текст текст текст текст текст текст текст текст текст текст текст текст текст текст текст текст текст текст текст текст текст текст текст текст текст текст.

\subparagraph{Подпункт 1} 
Текст текст текст текст текст текст текст текст текст текст текст текст текст текст текст текст текст текст текст текст текст текст текст текст текст текст текст.

\subparagraph{Подпункт 2} 
Текст текст текст текст текст текст текст текст текст текст текст текст текст текст текст текст текст текст текст текст текст текст текст текст текст текст текст.

\subparagraph{Подпункт 3} 
Текст текст текст текст текст текст текст текст текст текст текст текст текст текст текст текст текст текст текст текст текст текст текст текст текст текст текст.

\subparagraph{Пример одноуровнего перечисления} 

\begin{enumerate}
\item пункт перечисления;
\item пункт перечисления;
\item пункт перечисления;
\end{enumerate}

\subparagraph{Пример вложенного перечисления} 

%\begin{enumi}
%	\item пункт первый с большим количеством слов
%	\item здесь будет подпункт с немалым количеством слов
%	\begin{itemize}
%		\item подпункт первый с немалым количеством слов
%		\begin{itemize}
%			\item подпункт второй вложенности
%			\item еще один, коротенький пункт
%		\end{itemize}
%		\item подпункт первой степени вложенности
%	\end{itemize}
%	\item пункт
%\end{enumi}
%
\begin{enumerate}
\item пункт перечисления;
\item пункт перечисления;	
\begin{enumerate}%[label={\realasbuk*}, ref=\realasbuk*]
	\item пункт вложенного перечисления;
	\item пункт вложенного перечисления;
	\item пункт вложенного перечисления;
\end{enumerate}	
\item пункт перечисления;
\end{enumerate}







