\section{Введение}

\section{Основания для разработки}

\subsection{Документ, на основании которого ведется разработка}
Основанием для разработки данного приложения является приказ декана Факультета компьютерных наук И.В. Аржанцева № 2.3-02/1012-0 2 от 10.12.2018 «Об утверждении тем и руководителей курсовых работ студентов образовательной программы "Программная инженерия"».

\subsection{Наименование темы разработки}
Наименование темы разработки – «Кроссплатформенный облачный текстовый редактор "Notepad.Online"».

Условное обозначение темы разработки – «Notepad.Online».

Программа выполняется в рамках темы курсовой работы в соответствии с учебным планом подготовки бакалавров по направлению 09.03.04 «Программная инженерия» Национального исследовательского университета «Высшая школа экономики», факультет компьютерных наук, департамент программной инженерии.


%Данный пример иллюстрирует возможности доработанного класса espd.cls для создания программных документов по ГОСТ-19. Целью проделанной работы является автоматизация и упрощение написания документации к разрабатываемому программному обеспечению. Все побочные, оформительские сложности в данном случае оказываются полностью автоматизированы и не зависят, как в случае с word, от версии и т.д. \cite{espd001}  
