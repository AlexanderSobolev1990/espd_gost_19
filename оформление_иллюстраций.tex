\newpage\section{Оформление иллюстраций}

В данном разделе приводится пример оформления иллюстраций по п.~2.3 ГОСТ~19.106 \cite{gost19106}, где указано, что \textbf{подпись любой иллюстрации оформляется ключевым словом} <<Рис.~1>>, \textbf{а ссылка оформляется как }<<см.~рис.~1>>.

Иллюстрации, если их в документе более одной, нумеруют арабскими цифрами в пределах всего документа. В приложениях иллюстрации нумеруются в пределах каждого приложения аналогично как в основной части документа.

\illustration[][Тестовое изображение <<Лена>>][0.5]{Lenna}[fig:лена1]
\illustration[][Тестовое изображение <<Лена>>, вставленное еще раз для примера нумерации иллюстраций и уменьшенное в 2 раза][0.25]{Lenna}[fig:лена2]

В тексте документа возможно вставлять ссылки на иллюстрации, например так: см. \ref{fig:лена1} или см. \ref{fig:лена2}.

\newpage

Пример вставки двух изображений рядом:\\
{
\centering
\begin{tabular}[c]{ m{0.5\textwidth} m{0.5\textwidth} }		
	{
		\begin{minipage}[t]{0.45\textwidth}
			\centering
			\illustration[][Тестовое изображение <<Лена>> c очень длинной подписью][0.9]{Lenna}[fig:лена33]
		\end{minipage}
	} & {
		\begin{minipage}[t]{0.45\textwidth}
			\centering
			\illustration[][Тестовое изображение <<Лена>>][0.9]{Lenna}[fig:лена44]
		\end{minipage}
	} \\		
\end{tabular}
}

Возможно применить ссылку на рисунок в приложении: см.~\ref{fig:приложение1-лена1} приложения~\ref{atta:первое приложение}.
