\newpage\section{Оформление листингов исходного кода}

В данном разделе приводится пример оформления листингов исходного кода.

\subsection{Пример оформления иерархии вложенности проекта}

Данный пример получен при помощи команды \lstinline|tree| в linux.

\noindent
├── CMakeLists.txt\\
├── include\\
│   ├── compare.h\\
│   ├── lap.h\\
│   └── sparse.h\\
├── src\\
│   ├── laphungarian.cpp\\
│   ├── lapjvcdense.cpp\\
│   ├── lapjvcsparse.cpp\\
│   ├── lapmack.cpp\\
│   ├── lapseqextr.cpp\\
│   └── sparse.cpp\\
└── tree.txt\\

\subsection{Пример оформления листингов}

Листинг 1:

\begin{lstlisting}
//------------------------------------------------------------------------------
///
/// \file       sparse.h
/// \brief      Работа с разреженными матрицами
/// \details    Перевод матрицы из плотного представления в COO (Coordinate list), CSR (Compressed Sparse Row Yale
/// format), CSC (Compressed Sparse Column Yale format) вид.
/// \date       27.05.22 - создан
/// \author     Соболев А.А.
/// \addtogroup spml
/// \{
///

#ifndef SPML_SPARSE_H
#define SPML_SPARSE_H

// System includes:
#include <limits>
#include <armadillo>
#include <algorithm>

// SPML includes:
#include <compare.h>

namespace SPML /// Специальная библиотека программных модулей (СБ ПМ)
{
namespace Sparse /// Разреженные матрицы
{
//------------------------------------------------------------------------------
///
/// \brief Структура хранения матрицы в COO формате (Coordinate list)
/// \details Матрица A[n,m] (n - число строк, m - число столбцов)
///
struct CMatrixCOO
{
std::vector<double> coo_val;    ///< Вектор ненулевых элементов матрицы A[n,m] (n - число строк, m - число столбцов), размер nnz
std::vector<int> coo_row;       ///< Индексы строк ненулевых элементов
std::vector<int> coo_col;       ///< Индексы столбцов ненулевых элементов
};
} // end namespace Sparse
} // end namespace SPML
#endif // SPML_SPARSE_H
/// \}
\end{lstlisting}

\newpage
Листинг 2:

\begin{lstlisting}
//----------------------------------------------------------------------------------------------------------------------
///
/// \brief Ключ элемента Aij матрицы A в COO формате (Coordinate list)
/// \attention Оператор < перегружен для случая построчного хранения
///
struct CKeyCOO
{
public:
///
/// \brief i - индекс строки
///
int i() const { return i_; }

///
/// \brief j - индекс столбца
///
int j() const { return j_; }

///
/// \brief Конструктор по умолчанию
///
CKeyCOO() : i_( 0 ), j_( 0 ){}

///
/// \brief Параметрический конструктор
/// \param[in] i - индекс строки
/// \param[in] j - индекс столбца
///
CKeyCOO( int i, int j ) : i_( i ), j_( j ){}

bool operator <( CKeyCOO const& other ) const
{
if( ( this->i_ < other.i_ ) || ( ( this->i_ == other.i_ ) && ( this->j_ < other.j_ ) ) ) {
return true;
}
return false;
}

private:
int i_; ///< Индекс строки
int j_; ///< Индекс столбца
};
\end{lstlisting}



