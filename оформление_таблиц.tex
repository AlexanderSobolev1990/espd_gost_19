\newpage\section{Оформление таблиц}

В данном разделе приводится пример иерархии вложенности по п.~2.6 ГОСТ~19.106 \cite{gost19106}.

Оформление всегда следует вести при помощи класса \lstinline|longtable|, поскольку это дает возможность переноса длинной таблицы на следующую страницу.

\subsection{Простая маленькая таблица}

Простой пример маленькой таблицы c маленькими колонками, выровненными по центру, без каких-либо переполнений.
{\tabletextsize
% Всегда вставлять параметр [с] после longtable - выравнивание по центру страницы
\begin{longtable}[c]{| >{\centering}m{25mm} | >{\centering}m{25mm} | >{\centering}m{50mm} |}
	\caption{\normalsize Пример маленькой таблицы\hspace{25cm}} % \hspace{25cm} - хак, для того чтобы было выровнено по левому краю, иначе без головной боли не получилось
	\label{t:tab0} \\		
	\hline % горизонтальная линия
	колонка 1 & колонка 2 & колонка 3 \tabularnewline
	\hhline{|=|=|=|} % горизонтальная двойная линия
	111 & 222 & 333
	\tabularnewline\hline % горизонтальная линия
\end{longtable}
}

\subsection{Широкая таблица с длинными заголовками колонок}

Пример таблицы с длинными заголовками колонок
{\tabletextsize
\begin{longtable}[c]{| >{\raggedright}m{55mm} | >{\centering}m{55mm} | >{\raggedleft}m{55mm} |}
	\caption{\normalsize Пример таблицы c возможно очень длинным заголовком, который будет перенесен на вторую строчку\hspace{25cm}} % \hspace{25cm} - хак, для того чтобы было выровнено по левому краю, иначе без головной боли не получилось
	\label{t:tab1} \\		
	\hline % горизонтальная линия
	колонка 1 с очень длинным заголовком, просто капец каким длинным & колонка 2 по центру & колонка 3 по правому краю \tabularnewline
	\hhline{|=|=|=|} % горизонтальная двойная линия
	Содержание колонки 1 & Содержание колонки 2 & Содержание колонки 3, возможно очень длинное содержание, которое нормально отображается с переносом по словам
	\tabularnewline\hline % горизонтальная линия
\end{longtable}
}

Пример оформления ссылки на таблицу: см.~таблицу~\ref{t:tab1}.

\newpage
\subsection{Часто встречающая в документации таблица}

Пример таблицы, часто встречающийся в программной документации:

{\tabletextsize
\begin{longtable}[c]{| >{\centering}m{12mm} | >{\raggedright}m{53mm} | >{\centering}m{22mm} | >{\centering}m{15mm} | >{\raggedright}m{30mm} | >{\centering}m{18mm} |}
	%----------------------- преамбула ---------------------	
	\caption{\normalsize Пример таблицы\hspace{25cm}}
	\label{t:tab2} \\
	\hline
	\centering{Номер\\слова} & 
	\centering{Наименование информации} & 
	\centering{Условное\\обозначение} & 
	\centering{Размер-\\ность} & 
	\centering{Пределы\\изменения} & 
	\centering{Примеча-\\ние} \tabularnewline
	\hhline{|=|=|=|=|=|=|}
	\endfirsthead % Конец заголовка на 1 странице
	\multicolumn{6}{l}{Продолжение таблицы \thetable} \\ % 6 - число колонок, по-другому не получается
	\hline
	% Простой способ
%	Номер слова & Наименование информации & Усл.~об. & Размерн. & Пределы изменения & Примеч. \tabularnewline
	%
	% Способ, при котором раздельно формуруется выравнивание заголовка и контетнта
	\centering{Номер\\слова} & 
	\centering{Наименование информации} & 
	\centering{Условное\\обозначение} & 
	\centering{Размер-\\ность} & 
	\centering{Пределы\\изменения} & 
	\centering{Примеча-\\ние} \tabularnewline
	\hhline{|=|=|=|=|=|=|}
	\endhead
	\hline
	\multicolumn{6}{r}{\tabletextsize см. далее}
	\endfoot
	\hline
	\endlastfoot	
	%------------------- табличные данные ------------------

	1 & Контрольное слово & CW\textunderscore & б/р & \ndash & uint \tabularnewline\hline
	2 & Контрольное слово & CW\textunderscore & б/р & \ndash & uint \tabularnewline\hline
	3 & Контрольное слово & CW\textunderscore & б/р & \ndash & uint \tabularnewline\hline
	4 & Контрольное слово & CW\textunderscore & б/р & \ndash & uint \tabularnewline\hline
	5 & Контрольное слово & CW\textunderscore & б/р & \ndash & uint \tabularnewline\hline	
	6 & Контрольное слово & CW\textunderscore & б/р & \ndash & uint \tabularnewline\hline	
	7 & Контрольное слово & CW\textunderscore & б/р & \ndash & uint \tabularnewline\hline	
	8 & Контрольное слово & CW\textunderscore & б/р & \ndash & uint \tabularnewline\hline	
%
%	\newpage % Принудительный перенос на новую страницу
%
	9 & Контрольное слово & CW\textunderscore & б/р & \ndash & uint \tabularnewline\hline	
	10 & Контрольное слово & CW\textunderscore & б/р & \ndash & uint \tabularnewline\hline	
	11 & Контрольное слово & CW\textunderscore & б/р & \ndash & uint \tabularnewline\hline	
	12 & Контрольное слово & CW\textunderscore & б/р & \ndash & uint \tabularnewline\hline	
	13 & Контрольное слово & CW\textunderscore & б/р & \ndash & uint \tabularnewline\hline	
	14 & Контрольное слово & CW\textunderscore & б/р & \ndash & uint \tabularnewline\hline	
	15 & Контрольное слово & CW\textunderscore & б/р & \ndash & uint \tabularnewline\hline	
	16 & Контрольное слово & CW\textunderscore & б/р & \ndash & uint \tabularnewline\hline	
	17 & Контрольное слово & CW\textunderscore & б/р & \ndash & uint \tabularnewline\hline	
	18 & Контрольное слово & CW\textunderscore & б/р & \ndash & uint \tabularnewline\hline	
	19 & Контрольное слово & CW\textunderscore & б/р & \ndash & uint \tabularnewline\hline	
	20 & Контрольное слово & CW\textunderscore & б/р & \ndash & uint \tabularnewline\hline		
	20 & Контрольное слово & CW\textunderscore & б/р & \ndash & uint \tabularnewline\hline
	20 & Контрольное слово & CW\textunderscore & б/р & \ndash & uint \tabularnewline\hline
	20 & Контрольное слово & CW\textunderscore & б/р & \ndash & uint \tabularnewline\hline
	20 & Контрольное слово & CW\textunderscore & б/р & \ndash & uint \tabularnewline\hline
	20 & Контрольное слово & CW\textunderscore & б/р & \ndash & uint \tabularnewline\hline
	20 & Контрольное слово & CW\textunderscore & б/р & \ndash & uint \tabularnewline\hline
	20 & Контрольное слово & CW\textunderscore & б/р & \ndash & uint \tabularnewline\hline
	20 & Контрольное слово & CW\textunderscore & б/р & \ndash & uint \tabularnewline\hline
	20 & Контрольное слово & CW\textunderscore & б/р & \ndash & uint \tabularnewline\hline
	20 & Контрольное слово & CW\textunderscore & б/р & \ndash & uint \tabularnewline\hline
	20 & Контрольное слово & CW\textunderscore & б/р & \ndash & uint \tabularnewline\hline
	20 & Контрольное слово & CW\textunderscore & б/р & \ndash & uint \tabularnewline\hline
	20 & Контрольное слово & CW\textunderscore & б/р & \ndash & uint \tabularnewline\hline
	20 & Контрольное слово & CW\textunderscore & б/р & \ndash & uint \tabularnewline\hline
\end{longtable}
}

Пример оформления ссылки на таблицу: см.~таблицу~\ref{t:tab2}.

\subsection{Пример таблицы спецификации}

%\newcommand\xrowht[2][0]{\addstackgap[.5\dimexpr#2\relax]{\vphantom{#1}}}
{
	\renewcommand{\ULdepth}{1.8pt}\tabletextsize
	\begin{longtable}[c]{|>{\raggedright}m{74mm}|>{\raggedright}m{64mm}|>{\raggedright}m{25mm}|}
		%----------------------- преамбула ---------------------	
		\hline
		\centering{Обозначение} & 
		\centering{Наименование} & 
		\centering{Примечание} \xrowht[(5mm)]{5mm}\tabularnewline
		\hhline{|=|=|=|}\hline
		\endfirsthead % Конец заголовка на 1 странице
		\hline
		% Способ, при котором раздельно формуруется выравнивание заголовка и контетнта
		\centering{Обозначение} & 
		\centering{Наименование} & 
		\centering{Примечание} \xrowht[(5mm)]{5mm}\tabularnewline
		\hhline{|=|=|=|}
		\endhead
		\hline
		\endfoot
		\hline
		\endlastfoot	
		%------------------- табличные данные ------------------		
		~ & ~ & ~ \xrowht[(3.5mm)]{3.5mm}\tabularnewline\hline
		~ & \centering{\uline{Документация}} & ~ \xrowht[(3.5mm)]{3.5mm}\tabularnewline\hline
		~ & ~ & ~ \xrowht[(3.5mm)]{3.5mm}\tabularnewline\hline		

		1234567.12345-01-ЛУ & Лист утверждения & Размножать по указанию \xrowht[(3.5mm)]{3.5mm}\tabularnewline\hline
		1234567.12345-01-13 01 & Комплекс программ\\Самый лучший комплекс\\Описание программы & ~ \xrowht[(3.5mm)]{3.5mm}\tabularnewline\hline

		~ & ~ & ~ \xrowht[(3.5mm)]{3.5mm}\tabularnewline\hline
		~ & \centering{\uline{Компоненты}} & ~ \xrowht[(3.5mm)]{3.5mm}\tabularnewline\hline
		~ & ~ & ~ \xrowht[(3.5mm)]{3.5mm}\tabularnewline\hline

		1234567.12345-01 12 01-ДЭ & Программа\\Отображение информации\\Текст программы & Документ электронный \xrowht[(3.5mm)]{3.5mm}\tabularnewline\hline
		1234567.12345-01 12 01-УЛ & Информационно\sdash удостоверяющий лист & ~ \xrowht[(3.5mm)]{3.5mm}\tabularnewline\hline
		
		~ & ~ & ~ \xrowht[(3.5mm)]{3.5mm}\tabularnewline\hline
		~ & ~ & ~ \xrowht[(3.5mm)]{3.5mm}\tabularnewline\hline
		~ & ~ & ~ \xrowht[(3.5mm)]{3.5mm}\tabularnewline\hline
		~ & ~ & ~ \xrowht[(3.5mm)]{3.5mm}\tabularnewline\hline
		~ & ~ & ~ \xrowht[(3.5mm)]{3.5mm}\tabularnewline\hline
		~ & ~ & ~ \xrowht[(3.5mm)]{3.5mm}\tabularnewline\hline
		~ & ~ & ~ \xrowht[(3.5mm)]{3.5mm}\tabularnewline\hline
		~ & ~ & ~ \xrowht[(3.5mm)]{3.5mm}\tabularnewline\hline
		
		\newpage % Принудительный перенос на новую страницу
		
		пример если делать в каждой строке & ~ & ~ \xrowht[(3.5mm)]{3.5mm}\tabularnewline\hline
		~ & \centering{\uline{Документация}} & ~ \xrowht[(3.5mm)]{3.5mm}\tabularnewline\hline
		~ & ~ & ~ \xrowht[(3.5mm)]{3.5mm}\tabularnewline\hline		
		
		1234567.12345-01-ЛУ & Лист утверждения & Размножать \xrowht[(3.5mm)]{3.5mm}\tabularnewline\hline
		~ & ~ & по указанию  \xrowht[(3.5mm)]{3.5mm}\tabularnewline\hline		
		1234567.12345-01-13 01 & Комплекс программ & ~ \xrowht[(3.5mm)]{3.5mm}\tabularnewline\hline
		~ & Самый лучший комплекс & ~ \xrowht[(3.5mm)]{3.5mm}\tabularnewline\hline
		~ & Описание программы & ~ \xrowht[(3.5mm)]{3.5mm}\tabularnewline\hline
		
		~ & ~ & ~ \xrowht[(3.5mm)]{3.5mm}\tabularnewline\hline
		~ & \centering{\uline{Компоненты}} & ~ \xrowht[(3.5mm)]{3.5mm}\tabularnewline\hline
		~ & ~ & ~ \xrowht[(3.5mm)]{3.5mm}\tabularnewline\hline
		
		1234567.12345-01 12 01-ДЭ & Программа & Документ \xrowht[(3.5mm)]{3.5mm}\tabularnewline\hline
		~ & Отображение информации & электронный \xrowht[(3.5mm)]{3.5mm}\tabularnewline\hline
		~ & Текст программы & ~ \xrowht[(3.5mm)]{3.5mm}\tabularnewline\hline
		
		1234567.12345-01 12 01-УЛ & Информационно\sdash удостоверяющий & ~ \xrowht[(3.5mm)]{3.5mm}\tabularnewline\hline
		~ & лист & ~ \xrowht[(3.5mm)]{3.5mm}\tabularnewline\hline
		
		~ & ~ & ~ \xrowht[(3.5mm)]{3.5mm}\tabularnewline\hline
		~ & ~ & ~ \xrowht[(3.5mm)]{3.5mm}\tabularnewline\hline
		~ & ~ & ~ \xrowht[(3.5mm)]{3.5mm}\tabularnewline\hline
		~ & ~ & ~ \xrowht[(3.5mm)]{3.5mm}\tabularnewline\hline
		~ & ~ & ~ \xrowht[(3.5mm)]{3.5mm}\tabularnewline\hline
		~ & ~ & ~ \xrowht[(3.5mm)]{3.5mm}\tabularnewline\hline
		~ & ~ & ~ \xrowht[(3.5mm)]{3.5mm}\tabularnewline\hline
		~ & ~ & ~ \xrowht[(3.5mm)]{3.5mm}\tabularnewline\hline
		~ & ~ & ~ \xrowht[(3.5mm)]{3.5mm}\tabularnewline\hline
		~ & ~ & ~ \xrowht[(3.5mm)]{3.5mm}\tabularnewline\hline
		~ & ~ & ~ \xrowht[(3.5mm)]{3.5mm}\tabularnewline\hline
		~ & ~ & ~ \xrowht[(3.5mm)]{3.5mm}\tabularnewline\hline
		~ & ~ & ~ \xrowht[(3.5mm)]{3.5mm}\tabularnewline\hline
		~ & ~ & ~ \xrowht[(3.5mm)]{3.5mm}\tabularnewline\hline
		~ & ~ & ~ \xrowht[(3.5mm)]{3.5mm}\tabularnewline\hline
		~ & ~ & ~ \xrowht[(3.5mm)]{3.5mm}\tabularnewline\hline
		~ & ~ & ~ \xrowht[(3.5mm)]{3.5mm}\tabularnewline\hline				
	\end{longtable}
}

\subsection{Пример таблицы ведомости эксплуатационных документов}

%\newcommand\xrowht[2][0]{\addstackgap[.5\dimexpr#2\relax]{\vphantom{#1}}}
{
	\renewcommand{\ULdepth}{1.8pt}\tabletextsize
	\begin{longtable}[c]{|>{\raggedright}m{70mm}|>{\raggedright}m{60mm}|>{\centering}m{8mm}|>{\raggedright}m{24mm}|}
		%----------------------- преамбула ---------------------	
		\hline
		\centering{Обозначение} & 
		\centering{Наименование} & 
		\centering{Кол.\\экз.} & 
		\centering{Местона-\\хождение} \xrowht[(5mm)]{5mm}\tabularnewline
		\hhline{|=|=|=|=|}\hline
		\endfirsthead % Конец заголовка на 1 странице
		\hline
		% Способ, при котором раздельно формуруется выравнивание заголовка и контетнта
		\centering{Обозначение} & 
		\centering{Наименование} & 
		\centering{Кол.\\экз.} & 
		\centering{Местона-\\хождение} \xrowht[(5mm)]{5mm}\tabularnewline
		\hhline{|=|=|=|=|}
		\endhead
		\hline
		\endfoot
		\hline
		\endlastfoot	
		%------------------- табличные данные ------------------		
		~ & ~ & ~ & ~ \xrowht[(3.5mm)]{3.5mm}\tabularnewline\hline				
		~ & \centering{\uline{Документы на общую\\программу СПО}} & ~ & ~ \xrowht[(3.5mm)]{3.5mm}\tabularnewline\hline				
		~ & ~ & ~ & ~ \xrowht[(3.5mm)]{3.5mm}\tabularnewline\hline				
	
		1234567.12345-01 & Общая программа\\Специальное программное\\обеспечение\\Описание применения & 1 & ~ \xrowht[(3.5mm)]{3.5mm}\tabularnewline\hline

		~ & ~ & ~ & ~ \xrowht[(3.5mm)]{3.5mm}\tabularnewline\hline				
		~ & или аналогично таблице\\приведенной выше\\можно разбивать\\текст построчно,\\хотя мое 
		мнение\mdash это\\избыточно & ~ & ~ \xrowht[(3.5mm)]{3.5mm}\tabularnewline\hline				
		~ & ~ & ~ & ~ \xrowht[(3.5mm)]{3.5mm}\tabularnewline\hline
		~ & ~ & ~ & ~ \xrowht[(3.5mm)]{3.5mm}\tabularnewline\hline
		~ & ~ & ~ & ~ \xrowht[(3.5mm)]{3.5mm}\tabularnewline\hline
		~ & ~ & ~ & ~ \xrowht[(3.5mm)]{3.5mm}\tabularnewline\hline
		~ & ~ & ~ & ~ \xrowht[(3.5mm)]{3.5mm}\tabularnewline\hline
		~ & ~ & ~ & ~ \xrowht[(3.5mm)]{3.5mm}\tabularnewline\hline
		~ & ~ & ~ & ~ \xrowht[(3.5mm)]{3.5mm}\tabularnewline\hline
		~ & ~ & ~ & ~ \xrowht[(3.5mm)]{3.5mm}\tabularnewline\hline
		~ & ~ & ~ & ~ \xrowht[(3.5mm)]{3.5mm}\tabularnewline\hline
		~ & ~ & ~ & ~ \xrowht[(3.5mm)]{3.5mm}\tabularnewline\hline
		~ & ~ & ~ & ~ \xrowht[(3.5mm)]{3.5mm}\tabularnewline\hline
		~ & ~ & ~ & ~ \xrowht[(3.5mm)]{3.5mm}\tabularnewline\hline
		~ & ~ & ~ & ~ \xrowht[(3.5mm)]{3.5mm}\tabularnewline\hline
		~ & ~ & ~ & ~ \xrowht[(3.5mm)]{3.5mm}\tabularnewline\hline
		~ & ~ & ~ & ~ \xrowht[(3.5mm)]{3.5mm}\tabularnewline\hline
		~ & ~ & ~ & ~ \xrowht[(3.5mm)]{3.5mm}\tabularnewline\hline
		~ & ~ & ~ & ~ \xrowht[(3.5mm)]{3.5mm}\tabularnewline\hline
	\end{longtable}
}

%\newpage
%\subsection{Пример таблицы информационно-удостоверяющего листа}
%
%Данный лист используется в случае сдачи исходных текстов программ в виде *.zip архива с контрольной суммой (MD5).
%
%{
%	\renewcommand{\ULdepth}{1.8pt}\tabletextsize
%	\begin{longtable}[c]{|>{\centering}m{11mm}|>{\centering}m{50mm}|>{\centering}m{55mm}|>{\centering}m{20mm}|>{\centering}m{20mm}|}
%		%----------------------- преамбула ---------------------	
%		\hline
%		{Номер\\п/п} & 
%		{Обозначение документа} & 
%		{Наименование изделия,\\наименование документа} &
%		{Версия} &
%		{Номер\\последнего\\изменения} \xrowht[(5mm)]{5mm}\tabularnewline
%		\hhline{|=|=|=|=|=|}\hline
%		\endfirsthead % Конец заголовка на 1 странице
%		\hline
%		\endfoot
%		\hline
%		%------------------- табличные данные ------------------		
%		1 & АБВГ.ХХХХХХ.ХХЭСБ & Некое супер-изделие\\Электронная модель сборочной единицы & 2 & 1 \tabularnewline\hline
%	\end{longtable}
%}
%\vspace{-12pt}
%{
%	\begin{longtable}[c]{|>{\centering}m{20mm}|>{\raggedright}m{148mm}|}	
%		\hline
%		MD5 & 1234.5678.9012.2345 \tabularnewline\hline		
%	\end{longtable}	
%}
%\vspace{-12pt}
%{
%	\begin{longtable}[c]{|>{\centering}m{20mm}|>{\raggedright}m{148mm}|}	
%		\hline
%		MD5 & 1234.5678.9012.2345 \tabularnewline\hline		
%	\end{longtable}	
%}
%
