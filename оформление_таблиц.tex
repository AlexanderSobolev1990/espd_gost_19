\newpage\section{Оформление таблиц}

В данном разделе приводится пример иерархии вложенности по п.~2.6 ГОСТ~19.106 \cite{gost19106}.

Оформление всегда следует вести при помощи класса \lstinline|longtable|, поскольку это дает возможность переноса длинной таблицы на следующую страницу.

\subsection{Простая маленькая таблица}

Простой пример маленькой таблицы c маленькими колонками, выровненными по центру, без каких-либо переполнений.
{\tabletextsize
% Всегда вставлять параметр [с] после longtable - выравнивание по центру страницы
\begin{longtable}[c]{| >{\centering}m{25mm} | >{\centering}m{25mm} | >{\centering}m{50mm} |}
	\caption{\normalsize Пример маленькой таблицы\hspace{25cm}} % \hspace{25cm} - хак, для того чтобы было выровнено по левому краю, иначе без головной боли не получилось
	\label{t:tab0} \\		
	\hline % горизонтальная линия
	колонка 1 & колонка 2 & колонка 3 \tabularnewline
	\hhline{|=|=|=|} % горизонтальная двойная линия
	111 & 222 & 333
	\tabularnewline\hline % горизонтальная линия
\end{longtable}
}

\subsection{Широкая таблица с длинными заголовками колонок}

Пример таблицы с длинными заголовками колонок
{\tabletextsize
\begin{longtable}[c]{| >{\raggedright}m{55mm} | >{\centering}m{55mm} | >{\raggedleft}m{55mm} |}
	\caption{\normalsize Пример таблицы c возможно очень длинным заголовком, который будет перенесен на вторую строчку\hspace{25cm}} % \hspace{25cm} - хак, для того чтобы было выровнено по левому краю, иначе без головной боли не получилось
	\label{t:tab1} \\		
	\hline % горизонтальная линия
	колонка 1 с очень длинным заголовком, просто капец каким длинным & колонка 2 по центру & колонка 3 по правому краю \tabularnewline
	\hhline{|=|=|=|} % горизонтальная двойная линия
	Содержание колонки 1 & Содержание колонки 2 & Содержание колонки 3, возможно очень длинное содержание, которое нормально отображается с переносом по словам
	\tabularnewline\hline % горизонтальная линия
\end{longtable}
}

Пример оформления ссылки на таблицу: см.~таблицу~\ref{t:tab1}.

\newpage
\subsection{Часто встречающая в документации таблица}

Пример таблицы, часто встречающийся в программной документации:

{\tabletextsize
\begin{longtable}[c]{| >{\centering}m{12mm} | >{\raggedright}m{53mm} | >{\centering}m{20mm} | >{\centering}m{20mm} | >{\raggedright}m{30mm} | >{\centering}m{18mm} |}
	%----------------------- преамбула ---------------------	
	\caption{\normalsize Пример таблицы\hspace{25cm}}
	\label{t:tab2} \\
	\hline
	\multicolumn{1}{| >{\centering}m{12mm} |}{Номер слова} & 
	\multicolumn{1}{| >{\centering}m{53mm} |}{Наименование информации} & 
	\multicolumn{1}{| >{\centering}m{20mm} |}{Усл.~об.} & 
	\multicolumn{1}{| >{\centering}m{20mm} |}{Размерн.} & 
	\multicolumn{1}{| >{\centering}m{30mm} |}{Пределы изменения} & 
	\multicolumn{1}{| >{\centering}m{18mm} |}{Примеч.} \tabularnewline
	\hhline{|=|=|=|=|=|=|}
	\endfirsthead % Конец заголовка на 1 странице
	\multicolumn{6}{l}{Продолжение таблицы \thetable} \\ % 6 - число колонок, по-другому не получается
	\hline
	% Простой способ
%	Номер слова & Наименование информации & Усл.~об. & Размерн. & Пределы изменения & Примеч. \tabularnewline
	%
	% Способ, при котором раздельно формуруется выравнивание заголовка и контетнта
	\multicolumn{1}{| >{\centering}m{12mm} |}{Номер слова} & 
	\multicolumn{1}{| >{\centering}m{53mm} |}{Наименование информации} & 
	\multicolumn{1}{| >{\centering}m{20mm} |}{Усл.~об.} & 
	\multicolumn{1}{| >{\centering}m{20mm} |}{Размерн.} & 
	\multicolumn{1}{| >{\centering}m{30mm} |}{Пределы изменения} & 
	\multicolumn{1}{| >{\centering}m{18mm} |}{Примеч.} \tabularnewline
	\hhline{|=|=|=|=|=|=|}
	\endhead
	\hline
	\multicolumn{6}{r}{\tabletextsize см. далее}
	\endfoot
	\hline
	\endlastfoot	
	%------------------- табличные данные ------------------

	1 & Контрольное слово & CW\textunderscore & б/р & \ndash & uint \tabularnewline\hline
	2 & Контрольное слово & CW\textunderscore & б/р & \ndash & uint \tabularnewline\hline
	3 & Контрольное слово & CW\textunderscore & б/р & \ndash & uint \tabularnewline\hline
	4 & Контрольное слово & CW\textunderscore & б/р & \ndash & uint \tabularnewline\hline
	5 & Контрольное слово & CW\textunderscore & б/р & \ndash & uint \tabularnewline\hline	
	6 & Контрольное слово & CW\textunderscore & б/р & \ndash & uint \tabularnewline\hline	
	7 & Контрольное слово & CW\textunderscore & б/р & \ndash & uint \tabularnewline\hline	
	8 & Контрольное слово & CW\textunderscore & б/р & \ndash & uint \tabularnewline\hline	
%
%	\newpage % Принудительный перенос на новую страницу
%
	9 & Контрольное слово & CW\textunderscore & б/р & \ndash & uint \tabularnewline\hline	
	10 & Контрольное слово & CW\textunderscore & б/р & \ndash & uint \tabularnewline\hline	
	11 & Контрольное слово & CW\textunderscore & б/р & \ndash & uint \tabularnewline\hline	
	12 & Контрольное слово & CW\textunderscore & б/р & \ndash & uint \tabularnewline\hline	
	13 & Контрольное слово & CW\textunderscore & б/р & \ndash & uint \tabularnewline\hline	
	14 & Контрольное слово & CW\textunderscore & б/р & \ndash & uint \tabularnewline\hline	
	15 & Контрольное слово & CW\textunderscore & б/р & \ndash & uint \tabularnewline\hline	
	16 & Контрольное слово & CW\textunderscore & б/р & \ndash & uint \tabularnewline\hline	
	17 & Контрольное слово & CW\textunderscore & б/р & \ndash & uint \tabularnewline\hline	
	18 & Контрольное слово & CW\textunderscore & б/р & \ndash & uint \tabularnewline\hline	
	19 & Контрольное слово & CW\textunderscore & б/р & \ndash & uint \tabularnewline\hline	
	20 & Контрольное слово & CW\textunderscore & б/р & \ndash & uint \tabularnewline\hline		
	20 & Контрольное слово & CW\textunderscore & б/р & \ndash & uint \tabularnewline\hline
	20 & Контрольное слово & CW\textunderscore & б/р & \ndash & uint \tabularnewline\hline
	20 & Контрольное слово & CW\textunderscore & б/р & \ndash & uint \tabularnewline\hline
	20 & Контрольное слово & CW\textunderscore & б/р & \ndash & uint \tabularnewline\hline
	20 & Контрольное слово & CW\textunderscore & б/р & \ndash & uint \tabularnewline\hline
	20 & Контрольное слово & CW\textunderscore & б/р & \ndash & uint \tabularnewline\hline
	20 & Контрольное слово & CW\textunderscore & б/р & \ndash & uint \tabularnewline\hline
	20 & Контрольное слово & CW\textunderscore & б/р & \ndash & uint \tabularnewline\hline
	20 & Контрольное слово & CW\textunderscore & б/р & \ndash & uint \tabularnewline\hline
	20 & Контрольное слово & CW\textunderscore & б/р & \ndash & uint \tabularnewline\hline
	20 & Контрольное слово & CW\textunderscore & б/р & \ndash & uint \tabularnewline\hline
	20 & Контрольное слово & CW\textunderscore & б/р & \ndash & uint \tabularnewline\hline
	20 & Контрольное слово & CW\textunderscore & б/р & \ndash & uint \tabularnewline\hline
	20 & Контрольное слово & CW\textunderscore & б/р & \ndash & uint \tabularnewline\hline
\end{longtable}
}

Пример оформления ссылки на таблицу: см.~таблицу~\ref{t:tab2}.
