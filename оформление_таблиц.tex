\section{Оформление таблиц}

В данном разделе приводится пример иерархии вложенности по п.~2.6 ГОСТ~19.106 \cite{gost19106}.

Оформление всегда следует вести при помощи класса \lstinline|longtable|, поскольку это дает возможность переноса длинной таблицы на следующую страницу.

%\captionsetup[table]{skip=0pt,singlelinecheck=off}
%\LTcapwidth=\textwidth
%\setlength{\LTleft}{0pt}
\begin{longtable}{| >{\raggedright}m{55mm} | wc{55mm} | >{\raggedleft}m{55mm} |}
%	\caption{\mbox{\raggedright Пример организации длинной таблицы}} \\
	\caption{Пример организации длинной таблицы}
	\label{t:tab1} \\
	\hline
	колонка 1 очень  длинный заголовок колонки & 1-я строка & колонка 2 \tabularnewline
	\hline
	колонка \\ 1 & 2-я строка & колонка 3 2-я строка
	\tabularnewline\hline
\end{longtable}

Пример оформления ссылки на таблицу: см. \ref{t:tab1}.

%\begin{longtable}{|wc{55mm}|wc{55mm}|wc{55mm}|}
%	%----------------------- преамбула ---------------------
%	\caption{Пример организации длинной таблицы}
%	\label{t-LongTable}
%	\\% \toprule
%	\hline
%	Колонка 1 & Колонка 2 & Колонка 3 \tabularnewline 
%	\hline
%	\endfirsthead % Конец заголовка на 1 странице
%	\multicolumn{3}{r}{\tabletextsize продолжение табл.\thetable} \\ \hline
%	Колонка 1 & Колонка 2 & Колонка 3 \tabularnewline
%	\hline
%	\endhead
%	\hline
%	\multicolumn{3}{r}{\tabletextsize см. далее}
%	\endfoot
%	\hline
%	\endlastfoot
%	%------------------- табличные данные ------------------
%	текст первой ячейки \\больше длины колонки & 2 & 3 \\
%	\hline
%	1 & 2 & 3 \\
%	\hline
%	1 & 2 & 3 \\
%	\hline
%	1 & 2 & 3 \\
%
%	\newpage % Перенос на новую страницу
%	4 & 5 & 6 \\
%\end{longtable}