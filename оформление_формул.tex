\section{Оформление формул}

В данном разделе приводится пример оформления формул по п.~2.4 ГОСТ~19.106 \cite{gost19106}.

\subsection{Простые примеры}

\paragraph{Формула без присвоения порядкового номера}

Пример формулы, вставляемой в тексте без присвоения порядкового номера: формула квадратного многочлена: $f(x) = ax^2 + bx + c$, где $a$\mdash первый (старший) коэффициент, $b$\mdash второй (средний) коэффициент, $c$\mdash свободный член.

\paragraph{Формула с автоприсвоением порядкового номера}
Пример формул с присвоением порядкового номера и без удаления пробелов:
\begin{align}
	x = y+a \label{eq:формула 1}\\
	z = a+x \label{eq:формула 2}
\end{align}
\noindent где: $x$\ndash коэффициент 1; \\
\indent $z$\ndash коэффициент 2;\\
\indent $y$\ndash параметр 1;\\
\indent $a$\ndash параметр 2.

Пример ссылки на формулу: см. формулу (\ref{eq:формула 1}).

Пример формул с присвоением порядкового номера и с удалением пробелов:
\zerodisplayskips{
	\begin{align}
	x = y+a \label{eq:формула 3}\\
	z = a+x \label{eq:формула 4}
	\end{align}
}%
\noindent где: $x$\ndash коэффициент 1; \\
\indent $z$\ndash коэффициент 2;\\
\indent $y$\ndash параметр 1;\\
\indent $a$\ndash параметр 2.


