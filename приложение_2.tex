\newpage\attachment{Обязательное}{Пример приложения c номером и своими разделами}
\setcounter{equation}{0} % Сброс нумерации формул
\setcounter{table}{0} % Сброс нумерации таблиц
\setcounter{illustration}{0} % Сброс нумерации рисунков
\setcounter{section}{0} % Сброс нумерации разделов
%\setcounter{subsection}{0}

Текст нумерованного приложения. Формулы, рисунки, таблицы нумеруются заново в каждом приложении. Если приложение в документе одно, оно не нумеруется и это учитывает шаблон.

\section{Первый раздел приложения}

Пример начала раздела приложения.

\subsection{Пример формул в приложении}

Пример формул в приложении:
\zerodisplayskips{
	\begin{align}
	x = y+a \label{eq:формула приложения2_1}\\
	z = a+x \label{eq:формула приложения2_2}
	\end{align}
}%
\noindent где: $x$\ndash коэффициент 1; \\
\indent $z$\ndash коэффициент 2;\\
\indent $y$\ndash параметр 1;\\
\indent $a$\ndash параметр 2.

Пример ссылки на формулу приложения без номера: см. формулу \ref{eq:формула приложения2_1}.

\subsection{Пример рисунков в приложении}

Пример рисунков в приложении:
\illustration[][Тестовое изображение <<Лена>>][0.2]{Lenna}[img:приложение2-лена1]
\illustration[][Тестовое изображение <<Лена>>][0.2]{Lenna}[img:приложение2-лена2]

Пример ссылки на рисунки приложения без номера: см.~\ref{img:приложение2-лена1}.

\subsection{Пример таблицы в приложении}

Пример таблицы в приложении:

{\tabletextsize
	\begin{longtable}[c]{| >{\centering}m{12mm} | >{\raggedright}m{53mm} | >{\centering}m{20mm} | >{\centering}m{20mm} | >{\raggedright}m{30mm} | >{\centering}m{18mm} |}
		%----------------------- преамбула ---------------------	
		\caption{\normalsize Пример таблицы\hspace{25cm}}
		\label{t:таблица_приложения2} \\
		\hline
		\multicolumn{1}{| >{\centering}m{12mm} |}{Номер слова} & 
		\multicolumn{1}{| >{\centering}m{53mm} |}{Наименование информации} & 
		\multicolumn{1}{| >{\centering}m{20mm} |}{Усл.~об.} & 
		\multicolumn{1}{| >{\centering}m{20mm} |}{Размерн.} & 
		\multicolumn{1}{| >{\centering}m{30mm} |}{Пределы изменения} & 
		\multicolumn{1}{| >{\centering}m{18mm} |}{Примеч.} \tabularnewline
		\hhline{|=|=|=|=|=|=|}
		\endfirsthead % Конец заголовка на 1 странице
		\multicolumn{6}{l}{Продолжение таблицы \thetable} \\ % 6 - число колонок, по-другому не получается
		\hline
		% Простой способ
		%	Номер слова & Наименование информации & Усл.~об. & Размерн. & Пределы изменения & Примеч. \tabularnewline
		%
		% Способ, при котором раздельно формуруется выравнивание заголовка и контетнта
		\multicolumn{1}{| >{\centering}m{12mm} |}{Номер слова} & 
		\multicolumn{1}{| >{\centering}m{53mm} |}{Наименование информации} & 
		\multicolumn{1}{| >{\centering}m{20mm} |}{Усл.~об.} & 
		\multicolumn{1}{| >{\centering}m{20mm} |}{Размерн.} & 
		\multicolumn{1}{| >{\centering}m{30mm} |}{Пределы изменения} & 
		\multicolumn{1}{| >{\centering}m{18mm} |}{Примеч.} \tabularnewline
		\hhline{|=|=|=|=|=|=|}
		\endhead
		\hline
		\multicolumn{6}{r}{\tabletextsize см. далее}
		\endfoot
		\hline
		\endlastfoot	
		%------------------- табличные данные ------------------
		
		1 & Контрольное слово & CW\textunderscore & б/р & \ndash & uint \tabularnewline\hline
		2 & Контрольное слово & CW\textunderscore & б/р & \ndash & uint \tabularnewline\hline
		3 & Контрольное слово & CW\textunderscore & б/р & \ndash & uint \tabularnewline\hline
		4 & Контрольное слово & CW\textunderscore & б/р & \ndash & uint
	\end{longtable}
}

Пример оформления ссылки на таблицу приложения без номера: см.~таблицу~\ref{t:таблица_приложения2}.

\newpage
\section{Второй раздел приложения}

\subsection{Подраздел 1}
Текст текст текст текст текст текст текст текст текст текст текст текст текст текст текст текст текст текст текст текст текст текст текст текст текст текст текст.

\subsection{Подраздел 2}
Текст текст текст текст текст текст текст текст текст текст текст текст текст текст текст текст текст текст текст текст текст текст текст текст текст текст текст.

\paragraph{Пункт 1} 
Текст текст текст текст текст текст текст текст текст текст текст текст текст текст текст текст текст текст текст текст текст текст текст текст текст текст текст.

\paragraph{Пункт 2} 
Текст текст текст текст текст текст текст текст текст текст текст текст текст текст текст текст текст текст текст текст текст текст текст текст текст текст текст.

\subparagraph{Подпункт 1} 
Текст текст текст текст текст текст текст текст текст текст текст текст текст текст текст текст текст текст текст текст текст текст текст текст текст текст текст.

\subparagraph{Подпункт 2} 
Текст текст текст текст текст текст текст текст текст текст текст текст текст текст текст текст текст текст текст текст текст текст текст текст текст текст текст.
