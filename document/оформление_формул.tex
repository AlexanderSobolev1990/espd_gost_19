\newpage
\section{Оформление формул}

В данном разделе приводится пример оформления формул по п.~2.4 ГОСТ~19.106 \cite{gost19106}.

\subsection{Простые примеры}

\paragraph{Формула без присвоения порядкового номера}

Пример формулы, вставляемой в тексте без присвоения порядкового номера: формула квадратного многочлена: $f(x) = ax^2 + bx + c$, где $a$\ndash первый (старший) коэффициент, $b$\ndash второй (средний) коэффициент, $c$\ndash свободный член.

\paragraph{Формула с автоприсвоением порядкового номера и без удаления пробелов}
Пример формул с присвоением порядкового номера и без удаления пробелов:
\begin{align}
	x = y+a, \label{eq:формула 1}\\
	z = a+x, \label{eq:формула 2}
\end{align}

\formulalist{	
\item[где] $x$\ndash коэффициент, получаемый очень длинным способом, таким длинным, что его описание не влезает в одну строку и приходится делать перенос;
\item[] $z$\ndash коэффициент, получаемый очень длинным способом, таким длинным, что его описание не влезает в одну строку и приходится делать перенос;
\item[] $a$\ndash коэффициент, вычисляемый так, что надо делать вложенное уточнение как именно по формуле, полученной в трудах великих математиков:	
	\begin{align}
	a = b + c,
	\end{align} 

	\formulalist{
	\item[где] $b$\ndash коэффициент получаемый очень длинным;
	\item[] $c$\ndash коэффициент, получаемый очень длинным способом, таким длинным, что его описание не влезает в одну строку и приходится делать перенос;
	}
}

Пример ссылки на формулу: см. формулу (\ref{eq:формула 1}).

\newpage
\paragraph{Формула с автоприсвоением порядкового номера и с удалением пробелов}
Пример формул с присвоением порядкового номера и с удалением пробелов:
\formuladelspace
\begin{align}
x = y+a, \label{eq:формула 11}\\
z = a+x, \label{eq:формула 21}
\end{align}

\formulalist{
	\formuladelspace
	\item[где] $x$\ndash коэффициент, получаемый очень длинным способом, таким длинным, что его описание не влезает в одну строку и приходится делать перенос;
	\item[] $z$\ndash коэффициент, получаемый очень длинным способом, таким длинным, что его описание не влезает в одну строку и приходится делать перенос;
	\item[] $a$\ndash коэффициент, вычисляемый так, что надо делать вложенное уточнение как именно по формуле, полученной в трудах великих математиков:
	\formuladelspace
	\begin{align}
	a = b + c,
	\end{align} 	
	
	\formulalist{\formuladelspace
		\item[где] $b$\ndash коэффициент получаемый очень длинным;
		\item[] $c$\ndash коэффициент, получаемый очень длинным способом, таким длинным, что его описание не влезает в одну строку и приходится делать перенос;
	}
}

\subsection{Примеры различных математических знаков и символов}

Приведем несколько примеров.

\begin{enumerate}
%%%%%%%%%%%%%%%%%%%%%%%%%%%%%%%%%%%%%%%%%%%%%%%%%%%%%%%%%%%%%%%%%%%%%%%%%%%%%%%	
\item Индексы:
{\zerodisplayskips
	\begin{align}
	& A_1, \label{eq:ф1}\\
	& A_{\text{нижний индекс}}, \label{eq:ф2}\\
	& A^{\text{верхний индекс(он же - степень)}}, \label{eq:ф3}\\
	& A_{\text{нижний индекс}}^{\text{верхний индекс(он же - степень)}}, \label{eq:ф4} \\
	& P_0max. \label{eq:ф55}
	\end{align}	
}%
Отметим, что выравнивание в формулах (\ref{eq:ф1})\ndash (\ref{eq:ф4}) идёт по центру \textit{группы}.
%%%%%%%%%%%%%%%%%%%%%%%%%%%%%%%%%%%%%%%%%%%%%%%%%%%%%%%%%%%%%%%%%%%%%%%%%%%%%%%	
\item Знак градуса:
{\zerodisplayskips
\begin{equation}
X=5\degree.
\end{equation}	
}%
%%%%%%%%%%%%%%%%%%%%%%%%%%%%%%%%%%%%%%%%%%%%%%%%%%%%%%%%%%%%%%%%%%%%%%%%%%%%%%%	
\item Матрицы (жириный шрифт без наклона):
{\zerodisplayskips
\begin{equation}
	\matr{A},~\matr{B}_i^{-1},~\widetilde{\matr{C}}_i^{-1},~\overline{\boldsymbol{\sigma}}_i^{-1},~\matr{F\_Alarm}
\end{equation}	
}%
%%%%%%%%%%%%%%%%%%%%%%%%%%%%%%%%%%%%%%%%%%%%%%%%%%%%%%%%%%%%%%%%%%%%%%%%%%%%%%%
\item Верхние символы над одиночными буквами:	
{\zerodisplayskips
\begin{equation}
	\hat{\alpha_0},~\hat{A},~\tilde{B},~\bar{C},~\check{D},~\vec{E},~\hat{\mu}_{kl}^i.
\end{equation}
}%
%%%%%%%%%%%%%%%%%%%%%%%%%%%%%%%%%%%%%%%%%%%%%%%%%%%%%%%%%%%%%%%%%%%%%%%%%%%%%%%
\item Простановка фигурных скобок (обозначение элемента вектора, например):	
{\zerodisplayskips
	\begin{equation}
	\sigma_{ik}^{qw}\{ x \}=1.
	\end{equation}
}%
%%%%%%%%%%%%%%%%%%%%%%%%%%%%%%%%%%%%%%%%%%%%%%%%%%%%%%%%%%%%%%%%%%%%%%%%%%%%%%%
\item Верхние символы над многобуквенными переменными:	
{\zerodisplayskips
	\begin{equation}
	\widehat{\alpha},~\widehat{AB},~\widetilde{CD},~\overline{ABC},~\overrightarrow{DEF}.
	\end{equation}
}%
%%%%%%%%%%%%%%%%%%%%%%%%%%%%%%%%%%%%%%%%%%%%%%%%%%%%%%%%%%%%%%%%%%%%%%%%%%%%%%%
\item Некоторые греческие буквы с учетом русской традиции:	
{\zerodisplayskips
	\begin{equation}
	\alpha,\beta,\gamma,\phi,\epsilon,\theta. \label{eq:ф5}
	\end{equation}
}%
%%%%%%%%%%%%%%%%%%%%%%%%%%%%%%%%%%%%%%%%%%%%%%%%%%%%%%%%%%%%%%%%%%%%%%%%%%%%%%%
\item Сумма, умножение, деление, дробь:
{\zerodisplayskips
	\begin{align}
	\text{сумма:}\quad & A+B=C, \label{eq:ф6}\\
	\text{умножение:}\quad & A\times B=C, \label{eq:ф7}\\
	\text{деление через косую черту:}\quad & A/B=C, \label{eq:ф8}\\
	\text{дробь (решение квадратного уравнения):}\quad & x_{1,2}=\frac{-b\pm\sqrt{b^2-4ac}}{2a}, \label{eq:ф9}\\
	\text{дробь (решение квадратного уравнения):}\quad & x_{1,2}=\frac{-b\pm\sqrt{b^2-4ac}}{2a}. \label{eq:ф99}
	\end{align}
}%

Знак <<\&>> внутри формулы и конструкции \verb=\begin{align}...\end{align}= вызывает выравнивание по этому символу. 

Обратите внимание, повторная вставка формулы (\ref{eq:ф99}) вызывает автоматическое выравнивание по высоте.
%%%%%%%%%%%%%%%%%%%%%%%%%%%%%%%%%%%%%%%%%%%%%%%%%%%%%%%%%%%%%%%%%%%%%%%%%%%%%%
\item Производная и интеграл:
{\zerodisplayskips
	\begin{equation}	
	f'\quad f''\quad
	\dot{f}\quad \ddot{f} \quad
	\frac{d f}{d x}\quad
	\frac{\partial f}{\partial x}
	\int_0^{\infty}\quad
	\int\limits_0^{\infty}.\quad
	\label{eq:ф10}
	\end{equation}
}%
%%%%%%%%%%%%%%%%%%%%%%%%%%%%%%%%%%%%%%%%%%%%%%%%%%%%%%%%%%%%%%%%%%%%%%%%%%%%%%
\item Знак суммы:
{\zerodisplayskips
	\begin{equation}	
	\sum_{i=1}^n a_i,\quad
	\sum\nolimits_{i=1}^n b_i.
	\label{eq:ф11}
	\end{equation}
}
%%%%%%%%%%%%%%%%%%%%%%%%%%%%%%%%%%%%%%%%%%%%%%%%%%%%%%%%%%%%%%%%%%%%%%%%%%%%%%%
\item Перенос формул вручную c указанием места разделения и команды \verb=split=:
{\zerodisplayskips
	\begin{equation}	
	\begin{split}
	x&=1000+1100+{}\\
	 &+1200+1300.
	\end{split}
	\label{eq:ф12}
	\end{equation}
}
%%%%%%%%%%%%%%%%%%%%%%%%%%%%%%%%%%%%%%%%%%%%%%%%%%%%%%%%%%%%%%%%%%%%%%%%%%%%%%%
\item Cистема уравнений с фигурной скобкой (выравнивание по знаку <<=>>):
{%\zerodisplayskips
	\begin{equation}	
	\left\{
	\begin{aligned}
	x^2+y^2&=7 \\
	x+y & = 3. \\
	\end{aligned}
	\right.
	\end{equation}
}
%%%%%%%%%%%%%%%%%%%%%%%%%%%%%%%%%%%%%%%%%%%%%%%%%%%%%%%%%%%%%%%%%%%%%%%%%%%%%%%
\item Cистема уравнений с фигурной скобкой (выравнивание по левому краю):
{%\zerodisplayskips
	\begin{equation}	
	\left\{
	\begin{aligned}
	& x^2+y^2=7 \\
	& x+y=3.
	\end{aligned}
	\right.
	\end{equation}
}
%%%%%%%%%%%%%%%%%%%%%%%%%%%%%%%%%%%%%%%%%%%%%%%%%%%%%%%%%%%%%%%%%%%%%%%%%%%%%%%
\item Значение, зависящее от условий:
{%\zerodisplayskips
	\begin{equation}	
	|\sin(x)|=
	\begin{cases}
	\sin(x), & 0<x<\pi, \\
	-\sin(x), & \pi<x<2\pi.	
	\end{cases}
	\label{eq:ф13}
	\end{equation}
}
%%%%%%%%%%%%%%%%%%%%%%%%%%%%%%%%%%%%%%%%%%%%%%%%%%%%%%%%%%%%%%%%%%%%%%%%%%%%%%
\item Длина волны через частоту:
{\zerodisplayskips
	\begin{equation}	
	\lambda=C/(Fr \times 10^3).
	\end{equation}
}
%%%%%%%%%%%%%%%%%%%%%%%%%%%%%%%%%%%%%%%%%%%%%%%%%%%%%%%%%%%%%%%%%%%%%%%%%%%%%%
\item Пример очень длинной формулы с переносом на 2 строки:
{%\zerodisplayskips
	\begin{equation}
	\begin{split}	
	Vf_i&=X.V_i\times 0.5 \times (\cos((X.K_i - AzEndR_i)\times DgToRd) +{}\\
	    &+\cos((X.K_i - AzEndT_i)\times DgToRd).
	\label{eq:ф14}
	\end{split}
	\end{equation}
}
%%%%%%%%%%%%%%%%%%%%%%%%%%%%%%%%%%%%%%%%%%%%%%%%%%%%%%%%%%%%%%%%%%%%%%%%%%%%%%
\item Еще пример очень длинной формулы с переносом на 3 строки:
{%\zerodisplayskips
	\begin{equation}
	\begin{split}	
	ABCD&=X.V_i\times 0.5 \times (\cos((X.K_i - AzEndR_i)\times DgToRd) +{}\\
	&+\cos((X.K_i - AzEndT_i)\times DgToRd)+{}\\
	&+\cos((X.K_i - AzEndT_i)\times DgToRd).  
	\label{eq:ф15}
	\end{split}
	\end{equation}
}
%%%%%%%%%%%%%%%%%%%%%%%%%%%%%%%%%%%%%%%%%%%%%%%%%%%%%%%%%%%%%%%%%%%%%%%%%%%%%%
\item Пример автовыбора высоты скобок путем использования команд \verb=\left= и \verb=\right= соответственно:
{\zerodisplayskips1324
	\begin{equation}
	f(x)=1+\left(\frac{1}{1-x^{2}}
	\right)^3.
	\end{equation}
}
%%%%%%%%%%%%%%%%%%%%%%%%%%%%%%%%%%%%%%%%%%%%%%%%%%%%%%%%%%%%%%%%%%%%%%%%%%%%%%
\item Пример многоэтажной дроби:
{%\zerodisplayskips
	\begin{equation}
	X=\frac{\ln\left(\cfrac{A}{B}\right)\times \ln\left(\cfrac{C}{D}\right)}{\ln\left(\cfrac{E}{F} \right)\times \ln\left(\cfrac{G}{H} \right)}.
	\end{equation}
}
%%%%%%%%
\end{enumerate}	


