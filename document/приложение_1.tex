\newpage
\attachment{Обязательное}{Пример приложения с номером и без разделов}[atta:первое приложение]
\setcounter{equation}{0} % Сброс нумерации формул
\setcounter{table}{0} % Сброс нумерации таблиц
\setcounter{illustration}{0} % Сброс нумерации рисунков
%\setcounter{section}{0}
%\setcounter{subsection}{0}

Текст нумерованного приложения. Формулы, рисунки, таблицы нумеруются заново в каждом приложении. Если приложение в документе одно, оно не нумеруется и это учитывает шаблон.

Пример формул в приложении:
\formuladelspace
\begin{align}
x = y+a, \label{eq:формула приложения 1_1}\\
z = a+x, \label{eq:формула приложения 1_2}
\end{align}

\formulalist{\formuladelspace	
	\item[где] $x$\ndash коэффициент, получаемый очень длинным способом, таким длинным, что его описание не влезает в одну строку и приходится делать перенос;
	\item[] $z$\ndash коэффициент, получаемый очень длинным способом, таким длинным, что его описание не влезает в одну строку и приходится делать перенос;
	\item[] $a$\ndash коэффициент, вычисляемый так, что надо делать вложенное уточнение как именно по формуле, полученной в трудах великих математиков:	
	\formuladelspace
	\begin{align}
	a = b + c,
	\end{align} 	
	
	\formulalist{\formuladelspace		
		\item[где] $b$\ndash коэффициент получаемый очень длинным;
		\item[] $c$\ndash коэффициент, получаемый очень длинным способом, таким длинным, что его описание не влезает в одну строку и приходится делать перенос;
	}
}

Пример ссылки на формулу приложения без номера: см. формулу \ref{eq:формула приложения 1_1}.

Пример рисунков в приложении:
\illustration[][Тестовое изображение <<Лена>>][0.2]{Lenna}[fig:приложение1-лена1]
\illustration[][Тестовое изображение <<Лена>>][0.2]{Lenna}[fig:приложение1-лена2]

Пример ссылки на рисунки приложения без номера: см.~\ref{fig:приложение1-лена1}.

\newpage

Пример таблицы в приложении:

{\tabletextsize
	\begin{longtable}[c]{| >{\centering}m{12mm} | >{\raggedright}m{53mm} | >{\centering}m{22mm} | >{\centering}m{15mm} | >{\raggedright}m{33mm} | >{\centering}m{18mm} |}
		%----------------------- преамбула ---------------------	
		\caption{\normalsize Пример таблицы\hspace{25cm}}
		\label{t:таблица_приложения1} \\
		\hline
		\centering{Номер\\слова} & 
		\centering{Наименование информации} & 
		\centering{Условное\\обозначение} & 
		\centering{Размер-\\ность} & 
		\centering{Пределы\\изменения} & 
		\centering{Примеча-\\ние} \tabularnewline
		\hhline{|=|=|=|=|=|=|}
		\endfirsthead % Конец заголовка на 1 странице
		\multicolumn{6}{l}{Продолжение таблицы \thetable} \\ % 6 - число колонок, по-другому не получается
		\hline
		% Простой способ
		%	Номер слова & Наименование информации & Усл.~об. & Размерн. & Пределы изменения & Примеч. \tabularnewline
		%
		% Способ, при котором раздельно формуруется выравнивание заголовка и контетнта
		\centering{Номер\\слова} & 
		\centering{Наименование информации} & 
		\centering{Условное\\обозначение} & 
		\centering{Размер-\\ность} & 
		\centering{Пределы\\изменения} & 
		\centering{Примеча-\\ние} \tabularnewline
		\hhline{|=|=|=|=|=|=|}
		\endhead
		\hline
		\multicolumn{6}{r}{\tabletextsize см. далее}
		\endfoot
		\hline
		\endlastfoot	
		%------------------- табличные данные ------------------
		
		1 & Контрольное слово & CW\textunderscore & б/р & \ndash & uint \tabularnewline\hline
		2 & Контрольное слово & CW\textunderscore & б/р & \ndash & uint \tabularnewline\hline
		3 & Контрольное слово & CW\textunderscore & б/р & \ndash & uint \tabularnewline\hline
		4 & Контрольное слово & CW\textunderscore & б/р & \ndash & uint \tabularnewline\hline
	\end{longtable}
}

Пример оформления ссылки на таблицу приложения без номера: см.~таблицу~\ref{t:таблица_приложения1}.